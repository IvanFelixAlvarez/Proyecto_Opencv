\documentclass[11pt]{article}

\usepackage{listings} % Paquete para insertar códigos desde fichero

\usepackage[usenames,dvipsnames]{color} % Paquete para establecer colores por el nombre
\usepackage{mdframed} % Paquete para los recuadros de los códigos
\usepackage[utf8]{inputenc} % Para poner acentos y eñes directamente.
\usepackage{anysize} % Para establecer el margen
\usepackage{graphicx}
\usepackage{hyperref}
\usepackage[nottoc]{tocbibind}
\usepackage{framed, color}



\title{\textbf{Aplicaciones Micro-Robóticas\\
		       Proyecto de visión por computador con OpenCV}}
\author{\\Iván Félix Álvarez García\\\\\\
		\includegraphics[scale=0.45]{logo_uca}\\
		Universidad de Cádiz}
\date{30 de mayo de 2014}


\definecolor{gray}{rgb}{0.4,0.4,0.4}
\definecolor{darkblue}{rgb}{0.0,0.0,0.6}
\definecolor{cyan}{rgb}{0.0,0.6,0.6}

\lstset{
  basicstyle=\ttfamily,
  columns=fullflexible,
  showstringspaces=false,
  commentstyle=\color{gray}\upshape
}

\lstdefinelanguage{XML}
{
  morestring=[b]",
  morestring=[s]{>}{<},
  morecomment=[s]{<?}{?>},
  stringstyle=\color{black},
  identifierstyle=\color{darkblue},
  keywordstyle=\color{cyan},
  morekeywords={xmlns,version,type}% list your attributes here
}


\marginsize{2cm}{2cm}{1cm}{3cm} % Establecemos los margenes
		
\begin{document}
\maketitle
\newpage

\section{Introducción}

Este documento tiene como objetivo explicar el algoritmo de forma visual, sin entrar en detalle de implementación, para que, junto a las tomas de decisión que se han tenido que tomar, mostrar la trayectoria que he seguido a lo largo del proyecto.

\section{Captación}


\definecolor{shadecolor}{gray}{0.7}
\begin{shaded}
\textbf{Aclaración}. Se ha de concretar que estos elementos básicos se encuentran dentro de la secuencia, ya que al fin y al cabo deseamos que estas actividades sean ejecutadas de forma secuencial.
\end{shaded}



\end{document}
